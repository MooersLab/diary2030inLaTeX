% !TeX program = xetex
% !TeX options=--shell-escape
%%%%%%%%%%%%%%%%%%%% 750words2025.tex %%%%%%%%%%%%%%%%%%%%%%%%%%%%%
%
% sample root file for the chapters of your "monograph"
%
% Use this file as a template for your own input.
%
%%%%%%%%%%%%%%%% Springer-Verlag %%%%%%%%%%%%%%%%%%%%%%%%%%
% Intended LaTeX compiler: pdflatex
\documentclass[openany,envcountsame,envcountchap]{svmono}
                                    \usepackage[utf8]{inputenc}
\usepackage{rotating}
\usepackage{datetime}
\usepackage{appendix}
\usepackage[letterpaper, total={7in, 9in}]{geometry}
\usepackage[english]{babel}
\usepackage{minted}
\usepackage{listings}
\usepackage{caption}
\usepackage[caption=false]{subfig}
\usepackage{mdframed}
\usepackage{textcomp}
\usepackage{gensymb}
\usepackage{amsmath}
\usepackage{amssymb}
\usepackage{physics}
\usepackage{dsfont}
\usepackage{mathrsfs}
\usepackage{wasysym}
\usepackage{verbatim}
\usepackage[parfill]{parskip}
\usepackage{imakeidx}
\usepackage[xetex,plainpages=false]{hyperref}
\hypersetup{colorlinks=true, linkcolor=blue, citecolor=blue, urlcolor=blue, pdfborder={0 0 0}}
\makeindex[name=main,title=Subject Index,columns=2]
\makeindex[name=authors,title=Author Index,columns=2]
\makeindex[name=prompt,title=Writing Prompt Index,columns=1]
\usepackage{morewrites} % required when opening more than 16 files
\usepackage[acronym,toc]{glossaries}
\loadglsentries[type=acronym]{\jobname-acronyms.tex}
\loadglsentries{\jobname-glossary.tex}
\makeglossaries
\glsaddall
\newfloat{checklist}{H}{checklist}[chapter]
\floatname{checklist}{Checklist}
\makeatother
\usepackage[cc]{titlepic}
\usepackage{graphicx}
\usepackage{epigraph,varwidth}
\usepackage{epipart}
\renewcommand{\epigraphsize}{\small}
\setlength{\epigraphwidth}{0.6\textwidth}
\renewcommand{\textflush}{flushright}
\renewcommand{\sourceflush}{flushright}
\newcommand{\epitextfont}{\itshape}
\newcommand{\episourcefont}{\scshape}
\makeatletter
\newsavebox{\epi@textbox}
\newsavebox{\epi@sourcebox}
\newlength\epi@finalwidth
\renewcommand{\epigraph}[2]{%
\vspace{\beforeepigraphskip}
{\epigraphsize\begin{\epigraphflush}
\epi@finalwidth=\z@
\sbox\epi@textbox{%
\varwidth{\epigraphwidth}
\begin{\textflush}\epitextfont#1\end{\textflush}
\endvarwidth
}%
\epi@finalwidth=\wd\epi@textbox
\sbox\epi@sourcebox{%
\varwidth{\epigraphwidth}
\begin{\sourceflush}\episourcefont#2\end{\sourceflush}%
\endvarwidth
}%
\ifdim\wd\epi@sourcebox>\epi@finalwidth
\epi@finalwidth=\wd\epi@sourcebox
\fi
\leavevmode\vbox{
\hb@xt@\epi@finalwidth{\hfil\box\epi@textbox}
\vskip1.75ex
\hrule height \epigraphrule
\vskip.75ex
\hb@xt@\epi@finalwidth{\hfil\box\epi@sourcebox}
}%
\end{\epigraphflush}
\vspace{\afterepigraphskip}}}
\makeatother
\usepackage{booktabs}
\renewcommand{\lstlistingname}{Listing}
\renewcommand{\lstlistlistingname}{List of Listings}
\lstset{
basicstyle=\ttfamily\small,
frame=single,
breaklines=true,
numbers=left,
numberstyle=\tiny,
captionpos=b
}
\usepackage{xcolor}
\lstset{
language=Python,
keywordstyle=\color{blue},
commentstyle=\color{green!60!black},
stringstyle=\color{red}
}
\date{\today}
\author{Blaine Mooers}

\title{2027 words}
\hypersetup{
 pdfauthor={Blaine Mooers},
 pdftitle={2027 words},
 pdfkeywords={},
 pdfsubject={},
 pdfcreator={Emacs 30.1 (Org mode 9.7.11)},
 pdflang={English}}
\usepackage{natbib}

\begin{document}
%%%%%%%%%%%%%%%%%%%% 750words2030.tex %%%%%%%%%%%%%%%%%%%%%%%%%%%%%
%
% sample root file for the chapters of your "monograph"
%
% Use this file as a template for your own input.
%
%%%%%%%%%%%%%%%% Springer-Verlag %%%%%%%%%%%%%%%%%%%%%%%%%%


% % RECOMMENDED %%%%%%%%%%%%%%%%%%%%%%%%%%%%%%%%%%%%%%%%%%%%%%%%%%%
% \documentclass[envcountsame,envcountchap]{svmono}
%
% % choose options for [] as required from the list
% % in the Reference Guide, Sect. 2.2
% \usepackage[letterpaper, total={7in, 9in}]{geometry}
% \usepackage{makeidx}         % allows index generation
% \usepackage{graphicx}        % standard LaTeX graphics tool
%                              % when including figure files
% \usepackage{hyperref}  % Allows the use of \url{}
% \usepackage{multicol}        % used for the two-column index
% \usepackage[bottom]{footmisc}% places footnotes at page bottom
% \usepackage{amssymb}
% \usepackage{amsmath}
% \usepackage[normalem]{ulem} % enables strikethrough with \sout
% \usepackage{scrextend}
% \usepackage{minted}
% \usepackage{url}
% \makeatletter
% \g@addto@macro{\UrlBreaks}{\UrlOrds}
% \makeatother
%
% %%%%%%%%%%%%%%%%%%%%%%%%%%%%%%%%%%%%%%%%%%%%%%%%%%%%%%%%%%%%%%%%%%%%%
% \usepackage{tcolorbox}
% \tcbuselibrary{minted,skins}
% \newtcblisting{bashcode}[1][]{
%   listing engine=minted,
%   colback=bg,
%   colframe=black!70,
%   listing only,
%   minted style=colorful,
%   minted language=bash,
%   minted options={linenos=true,numbersep=3mm,texcl=true,#1},
%   left=5mm,enhanced,
%   overlay={\begin{tcbclipinterior}\fill[black!25] (frame.south west)
%             rectangle ([xshift=5mm]frame.north west);\end{tcbclipinterior}}
% }
% \definecolor{bg}{rgb}{0.97,0.97,0.97}
% % see the list of further useful packages
% % in the Reference Guide, Sects. 2.3, 3.1-3.3
% %%%%%%%%%%%%%%%%%%%%%%%%%%%%%%%%%%%%%%%%%%%%%%%%%%%%%%%%%%%%%%%%%%%%%
% % Make Verbatim environment
%
%
%
%
%
% %%%%%%%%%%%%%%%%%%%%%%%%%%%%%%%%%%%%%%%%%%%%%%%%%%%%%%%%%%%%%%%%%%%%%
%
% \usepackage{filecontents}% http://ctan.org/pkg/filecontents
% \usepackage{listings}% http://ctan.org/pkg/listings
% \addtokomafont{labelinglabel}{\sffamily\bfseries}
%
% % etc.
% % see the list of further useful packages
% % in the Reference Guide, Sects. 2.3, 3.1-3.3
%
% \makeindex             % used for the subject index
%                        % please use the style svind.ist with
%                        % your makeindex program
% %%%%%%%%%%%%%%%%%%%%%%%%%%%%%%%%%%%%%%%%%%%%%%%%%%%%%%%%%%%%%%%%%%%%%
%\begin{document}

\maketitle

\frontmatter%%%%%%%%%%%%%%%%%%%%%%%%%%%%%%%%%%%%%%%%%%%%%%%%%%%%%%

%!TEX root = ../main.tex  
%!TEX options=--shell-escape  

%%%%%%%%%%%%%%%%%%%%%% pref.tex %%%%%%%%%%%%%%%%%%%%%%%%%%%%%%%%%%%%%
%
% sample preface
%
% Use this file as a template for your own input.
%
%%%%%%%%%%%%%%%%%%%%%%%% Springer-Verlag %%%%%%%%%%%%%%%%%%%%%%%%%%

\preface
\index{preface}
%% Please write your preface here
Here come the golden words


%% Please "sign" your preface
\vspace{1cm}
\begin{flushright}\noindent
place(s),\hfill {\it First name  Surname}\\
month year\hfill {\it First name  Surname}\\
\end{flushright}


\section*{Known issues with typesetting this document}
\index{750words}

\subsection*{Author index generation}
The documentclass does not support author indices directly.
The author index is formatted using the listofauthors command.
This command is defined in the Preamble. 
I found this command online.
I tried to add the multicols environment.
The rightmost column overlapped with the header line on the first page of the index.
Hiding this line might solve the problem.

\subsection*{Underscores and the xurl package}
This package cannot handle underscores in URLs.
I obtained a memory error.
The solution was to comment out the xrul package.
The solution was provided by Dr. LainTze Lim of Overleaf.

\subsection*{Do not overuse labels}
Add labels only when you need them.
They will be unreferenced and will trigger many warnings.
In addition, do not use labels in templates that write TeX files unless you write unique labels.
Otherwise, you will get a warning about multiple identical labels.
It is the situation with the month.tex files.
All twelve had an intro label.


RAM may limit 750words to one document per year.



\tableofcontents

\mainmatter%%%%%%%%%%%%%%%%%%%%%%%%%%%%%%%%%%%%%%%%%%%%%%%%%%%%%%%
\include{part}
\include{./Content/January/January2030}
%!TeX options=--shell-escape
%%%%%%%%%%%%%%%%%%%%% chapter.tex %%%%%%%%%%%%%%%%%%%%%%%%%%%%%%%%%
%
% sample chapter
%
% Use this file as a template for your own input.
%
%%%%%%%%%%%%%%%%%%%%%%%% Springer-Verlag %%%%%%%%%%%%%%%%%%%%%%%%%%

\chapter{February 2030}
\label{february2030} % Always give a unique label

\section{February 1}
\input{./Content/February/1February2030}
\section{February 2}
\input{./Content/February/2February2030}
\section{February 3}
\input{./Content/February/3February2030}
\section{February 4}
\input{./Content/February/4February2030}
\section{February 5}
\input{./Content/February/5February2030}
\section{February 6}
\input{./Content/February/6February2030}
\section{February 7}
\input{./Content/February/7February2030}
\section{February 8}
\input{./Content/February/8February2030}
\section{February 9}
\input{./Content/February/9February2030}
\section{February 10}
\input{./Content/February/10February2030}
\section{February 11}
\input{./Content/February/11February2030}
\section{February 12}
\input{./Content/February/12February2030}
\section{February 13}
\input{./Content/February/13February2030}
\section{February 14}
\input{./Content/February/14February2030}
\section{February 15}
\input{./Content/February/15February2030}
\section{February 16}
\input{./Content/February/16February2030}
\section{February 17}
\input{./Content/February/17February2030}
\section{February 18}
\input{./Content/February/18February2030}
\section{February 19}
\input{./Content/February/19February2030}
\section{February 20}
\input{./Content/February/20February2030}
\section{February 21}
\input{./Content/February/21February2030}
\section{February 22}
\input{./Content/February/22February2030}
\section{February 23}
\input{./Content/February/23February2030}
\section{February 24}
\input{./Content/February/24February2030}
\section{February 25}
\input{./Content/February/25February2030}
\section{February 26}
\input{./Content/February/26February2030}
\section{February 27}
\input{./Content/February/27February2030}
\section{February 28}
\input{./Content/February/28February2030}

%!TeX options=--shell-escape
%%%%%%%%%%%%%%%%%%%%% chapter.tex %%%%%%%%%%%%%%%%%%%%%%%%%%%%%%%%%
%
% sample chapter
%
% Use this file as a template for your own input.
%
%%%%%%%%%%%%%%%%%%%%%%%% Springer-Verlag %%%%%%%%%%%%%%%%%%%%%%%%%%

\chapter{March 2030}
\label{march2030} % Always give a unique label

\section{March 1}
\input{./Content/March/1March2030}
\section{March 2}
\input{./Content/March/2March2030}
\section{March 3}
\input{./Content/March/3March2030}
\section{March 4}
\input{./Content/March/4March2030}
\section{March 5}
\input{./Content/March/5March2030}
\section{March 6}
\input{./Content/March/6March2030}
\section{March 7}
\input{./Content/March/7March2030}
\section{March 8}
\input{./Content/March/8March2030}
\section{March 9}
\input{./Content/March/9March2030}
\section{March 10}
\input{./Content/March/10March2030}
\section{March 11}
\input{./Content/March/11March2030}
\section{March 12}
\input{./Content/March/12March2030}
\section{March 13}
\input{./Content/March/13March2030}
\section{March 14}
\input{./Content/March/14March2030}
\section{March 15}
\input{./Content/March/15March2030}
\section{March 16}
\input{./Content/March/16March2030}
\section{March 17}
\input{./Content/March/17March2030}
\section{March 18}
\input{./Content/March/18March2030}
\section{March 19}
\input{./Content/March/19March2030}
\section{March 20}
\input{./Content/March/20March2030}
\section{March 21}
\input{./Content/March/21March2030}
\section{March 22}
\input{./Content/March/22March2030}
\section{March 23}
\input{./Content/March/23March2030}
\section{March 24}
\input{./Content/March/24March2030}
\section{March 25}
\input{./Content/March/25March2030}
\section{March 26}
\input{./Content/March/26March2030}
\section{March 27}
\input{./Content/March/27March2030}
\section{March 28}
\input{./Content/March/28March2030}
\section{March 29}
\input{./Content/March/29March2030}
\section{March 30}
\input{./Content/March/30March2030}
\section{March 31}
\input{./Content/March/31March2030}

%!TeX options=--shell-escape
%%%%%%%%%%%%%%%%%%%%% chapter.tex %%%%%%%%%%%%%%%%%%%%%%%%%%%%%%%%%
%
% sample chapter
%
% Use this file as a template for your own input.
%
%%%%%%%%%%%%%%%%%%%%%%%% Springer-Verlag %%%%%%%%%%%%%%%%%%%%%%%%%%

\chapter{April 2030}
\label{april2030} % Always give a unique label

\section{April 1}
\input{./Content/April/1April2030}
\section{April 2}
\input{./Content/April/2April2030}
\section{April 3}
\input{./Content/April/3April2030}
\section{April 4}
\input{./Content/April/4April2030}
\section{April 5}
\input{./Content/April/5April2030}
\section{April 6}
\input{./Content/April/6April2030}
\section{April 7}
\input{./Content/April/7April2030}
\section{April 8}
\input{./Content/April/8April2030}
\section{April 9}
\input{./Content/April/9April2030}
\section{April 10}
\input{./Content/April/10April2030}
\section{April 11}
\input{./Content/April/11April2030}
\section{April 12}
\input{./Content/April/12April2030}
\section{April 13}
\input{./Content/April/13April2030}
\section{April 14}
\input{./Content/April/14April2030}
\section{April 15}
\input{./Content/April/15April2030}
\section{April 16}
\input{./Content/April/16April2030}
\section{April 17}
\input{./Content/April/17April2030}
\section{April 18}
\input{./Content/April/18April2030}
\section{April 19}
\input{./Content/April/19April2030}
\section{April 20}
\input{./Content/April/20April2030}
\section{April 21}
\input{./Content/April/21April2030}
\section{April 22}
\input{./Content/April/22April2030}
\section{April 23}
\input{./Content/April/23April2030}
\section{April 24}
\input{./Content/April/24April2030}
\section{April 25}
\input{./Content/April/25April2030}
\section{April 26}
\input{./Content/April/26April2030}
\section{April 27}
\input{./Content/April/27April2030}
\section{April 28}
\input{./Content/April/28April2030}
\section{April 29}
\input{./Content/April/29April2030}
\section{April 30}
\input{./Content/April/30April2030}

\include{./Content/May/May2030}
\include{./Content/June/June2030}
%!TeX options=--shell-escape
%%%%%%%%%%%%%%%%%%%%% chapter.tex %%%%%%%%%%%%%%%%%%%%%%%%%%%%%%%%%
%
% sample chapter
%
% Use this file as a template for your own input.
%
%%%%%%%%%%%%%%%%%%%%%%%% Springer-Verlag %%%%%%%%%%%%%%%%%%%%%%%%%%

\chapter{July 2030}
\label{july2030} % Always give a unique label

\section{July 1}
\input{./Content/July/1July2030}
\section{July 2}
\input{./Content/July/2July2030}
\section{July 3}
\input{./Content/July/3July2030}
\section{July 4}
\input{./Content/July/4July2030}
\section{July 5}
\input{./Content/July/5July2030}
\section{July 6}
\input{./Content/July/6July2030}
\section{July 7}
\input{./Content/July/7July2030}
\section{July 8}
\input{./Content/July/8July2030}
\section{July 9}
\input{./Content/July/9July2030}
\section{July 10}
\input{./Content/July/10July2030}
\section{July 11}
\input{./Content/July/11July2030}
\section{July 12}
\input{./Content/July/12July2030}
\section{July 13}
\input{./Content/July/13July2030}
\section{July 14}
\input{./Content/July/14July2030}
\section{July 15}
\input{./Content/July/15July2030}
\section{July 16}
\input{./Content/July/16July2030}
\section{July 17}
\input{./Content/July/17July2030}
\section{July 18}
\input{./Content/July/18July2030}
\section{July 19}
\input{./Content/July/19July2030}
\section{July 20}
\input{./Content/July/20July2030}
\section{July 21}
\input{./Content/July/21July2030}
\section{July 22}
\input{./Content/July/22July2030}
\section{July 23}
\input{./Content/July/23July2030}
\section{July 24}
\input{./Content/July/24July2030}
\section{July 25}
\input{./Content/July/25July2030}
\section{July 26}
\input{./Content/July/26July2030}
\section{July 27}
\input{./Content/July/27July2030}
\section{July 28}
\input{./Content/July/28July2030}
\section{July 29}
\input{./Content/July/29July2030}
\section{July 30}
\input{./Content/July/30July2030}
\section{July 31}
\input{./Content/July/31July2030}

\include{./Content/August/August2030}
%!TeX options=--shell-escape
%%%%%%%%%%%%%%%%%%%%% chapter.tex %%%%%%%%%%%%%%%%%%%%%%%%%%%%%%%%%
%
% sample chapter
%
% Use this file as a template for your own input.
%
%%%%%%%%%%%%%%%%%%%%%%%% Springer-Verlag %%%%%%%%%%%%%%%%%%%%%%%%%%

\chapter{September 2030}
\label{september2030} % Always give a unique label

\section{September 1}
\input{./Content/September/1September2030}
\section{September 2}
\input{./Content/September/2September2030}
\section{September 3}
\input{./Content/September/3September2030}
\section{September 4}
\input{./Content/September/4September2030}
\section{September 5}
\input{./Content/September/5September2030}
\section{September 6}
\input{./Content/September/6September2030}
\section{September 7}
\input{./Content/September/7September2030}
\section{September 8}
\input{./Content/September/8September2030}
\section{September 9}
\input{./Content/September/9September2030}
\section{September 10}
\input{./Content/September/10September2030}
\section{September 11}
\input{./Content/September/11September2030}
\section{September 12}
\input{./Content/September/12September2030}
\section{September 13}
\input{./Content/September/13September2030}
\section{September 14}
\input{./Content/September/14September2030}
\section{September 15}
\input{./Content/September/15September2030}
\section{September 16}
\input{./Content/September/16September2030}
\section{September 17}
\input{./Content/September/17September2030}
\section{September 18}
\input{./Content/September/18September2030}
\section{September 19}
\input{./Content/September/19September2030}
\section{September 20}
\input{./Content/September/20September2030}
\section{September 21}
\input{./Content/September/21September2030}
\section{September 22}
\input{./Content/September/22September2030}
\section{September 23}
\input{./Content/September/23September2030}
\section{September 24}
\input{./Content/September/24September2030}
\section{September 25}
\input{./Content/September/25September2030}
\section{September 26}
\input{./Content/September/26September2030}
\section{September 27}
\input{./Content/September/27September2030}
\section{September 28}
\input{./Content/September/28September2030}
\section{September 29}
\input{./Content/September/29September2030}
\section{September 30}
\input{./Content/September/30September2030}

%!TeX options=--shell-escape
%%%%%%%%%%%%%%%%%%%%% chapter.tex %%%%%%%%%%%%%%%%%%%%%%%%%%%%%%%%%
%
% sample chapter
%
% Use this file as a template for your own input.
%
%%%%%%%%%%%%%%%%%%%%%%%% Springer-Verlag %%%%%%%%%%%%%%%%%%%%%%%%%%

\chapter{October 2030}
\label{october2030} % Always give a unique label

\section{October 1}
\input{./Content/October/1October2030}
\section{October 2}
\input{./Content/October/2October2030}
\section{October 3}
\input{./Content/October/3October2030}
\section{October 4}
\input{./Content/October/4October2030}
\section{October 5}
\input{./Content/October/5October2030}
\section{October 6}
\input{./Content/October/6October2030}
\section{October 7}
\input{./Content/October/7October2030}
\section{October 8}
\input{./Content/October/8October2030}
\section{October 9}
\input{./Content/October/9October2030}
\section{October 10}
\input{./Content/October/10October2030}
\section{October 11}
\input{./Content/October/11October2030}
\section{October 12}
\input{./Content/October/12October2030}
\section{October 13}
\input{./Content/October/13October2030}
\section{October 14}
\input{./Content/October/14October2030}
\section{October 15}
\input{./Content/October/15October2030}
\section{October 16}
\input{./Content/October/16October2030}
\section{October 17}
\input{./Content/October/17October2030}
\section{October 18}
\input{./Content/October/18October2030}
\section{October 19}
\input{./Content/October/19October2030}
\section{October 20}
\input{./Content/October/20October2030}
\section{October 21}
\input{./Content/October/21October2030}
\section{October 22}
\input{./Content/October/22October2030}
\section{October 23}
\input{./Content/October/23October2030}
\section{October 24}
\input{./Content/October/24October2030}
\section{October 25}
\input{./Content/October/25October2030}
\section{October 26}
\input{./Content/October/26October2030}
\section{October 27}
\input{./Content/October/27October2030}
\section{October 28}
\input{./Content/October/28October2030}
\section{October 29}
\input{./Content/October/29October2030}
\section{October 30}
\input{./Content/October/30October2030}
\section{October 31}
\input{./Content/October/31October2030}

\include{./Content/November/November2030}
%!TeX options=--shell-escape
%%%%%%%%%%%%%%%%%%%%% chapter.tex %%%%%%%%%%%%%%%%%%%%%%%%%%%%%%%%%
%
% sample chapter
%
% Use this file as a template for your own input.
%
%%%%%%%%%%%%%%%%%%%%%%%% Springer-Verlag %%%%%%%%%%%%%%%%%%%%%%%%%%

\chapter{December 2030}
\label{december2030} % Always give a unique label

\section{December 1}
\input{./Content/December/1December2030}
\section{December 2}
\input{./Content/December/2December2030}
\section{December 3}
\input{./Content/December/3December2030}
\section{December 4}
\input{./Content/December/4December2030}
\section{December 5}
\input{./Content/December/5December2030}
\section{December 6}
\input{./Content/December/6December2030}
\section{December 7}
\input{./Content/December/7December2030}
\section{December 8}
\input{./Content/December/8December2030}
\section{December 9}
\input{./Content/December/9December2030}
\section{December 10}
\input{./Content/December/10December2030}
\section{December 11}
\input{./Content/December/11December2030}
\section{December 12}
\input{./Content/December/12December2030}
\section{December 13}
\input{./Content/December/13December2030}
\section{December 14}
\input{./Content/December/14December2030}
\section{December 15}
\input{./Content/December/15December2030}
\section{December 16}
\input{./Content/December/16December2030}
\section{December 17}
\input{./Content/December/17December2030}
\section{December 18}
\input{./Content/December/18December2030}
\section{December 19}
\input{./Content/December/19December2030}
\section{December 20}
\input{./Content/December/20December2030}
\section{December 21}
\input{./Content/December/21December2030}
\section{December 22}
\input{./Content/December/22December2030}
\section{December 23}
\input{./Content/December/23December2030}
\section{December 24}
\input{./Content/December/24December2030}
\section{December 25}
\input{./Content/December/25December2030}
\section{December 26}
\input{./Content/December/26December2030}
\section{December 27}
\input{./Content/December/27December2030}
\section{December 28}
\input{./Content/December/28December2030}
\section{December 29}
\input{./Content/December/29December2030}
\section{December 30}
\input{./Content/December/30December2030}
\section{December 31}
\input{./Content/December/31December2030}

%\include{chapter}
\appendix
%\include{./Appendices/ProblemsInMolecularGraphics}
%\include{./Appendices/WritingProjects}
%\include{./Appendices/TimeExpended}
%\include{./Appendices/sequence1}
%\include{./Appendices/sequence2}
%\backmatter%%%%%%%%%%%%%%%%%%%%%%%%%%%%%%%%%%%%%%%%%%%%%%%%%%%%%%%
%\include{solutions}
%\include{references}
\printindex
%\end{document}

%%%%%%%%%%%%%%%%%%%%%%%%%%%%%%%%%%%%%%%%%%%%%%%%%%%%%%%%%%%%%%%%%%%%%%

\end{document}
%%% Local Variables:
%%% mode: LaTeX
%%% TeX-master: t
%%% End:
